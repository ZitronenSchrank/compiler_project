\section{Hauptkonzepte}
Im folgenden Kapitel werden die wenigen Konzepte der entwickelten Sprache vorgestellt und deren Funktion erläutert. Mit diesem Wissen ist es anschließend möglich eigene Programme mithilfe der while-Programmiersprache zu schreiben.

\subsection{Kommentare}
Wie aus vielen anderen Programmiersprachen bekannt, kann der Programmcode mithilfe von zwei führenden Schrägstrichen  kommentiert werden.

\subsection{Variablen}
Variablendeklarationen werden mit dem Schlüsselwort \textbf{var} gekennzeichnet. Darauf folgt der Name der Variable. Anders als in anderen Programmiersprachen ist es hier notwendig der Variable bei der Deklaration auch einen Wert zuzuweisen. Eine Zuweisung erfolgt mit einem Doppelpunkt gefolgt von einem Gleichheitszeichen (:=). Einer Variable können Konstanten, Variablen oder Funktionsaufrufe zugewiesen werden. Nachdem eine Variable erzeugt wurde kann ihr ein neuer Wert zugewiesen werden oder sie in Funktionsaufrufen genutzt werden.

\begin{lstlisting}[language=c, caption=Variablennutzung in While, label={lst:while-var-defdec}]
var r0 := 0; // Erzeugt eine Variable mit dem Namen 'r0' und dem Wert 0
var r1 := r0; // Erzeugt eine Variable mit dem Namen 'r1' und dem Wert von r0
r0 := 5; // r0 bekommt einen neuen Wert
\end{lstlisting}

\subsection{Pred und Succ}
\textbf{pred} und \textbf{succ} sind zwei Funktionen, welche die Sprache zur Verfügung stellt und den Wert von einer Variable direkt manipulieren können. Nach dem jeweiligen Schlüsselwort folgt der Name einer Variable innerhalb von zwei runden Klammern. \textbf{pred} steht für \enquote{predecessor} und wird genutzt um den Wert einer Variable um eine Stelle zu verringern. Analog dazu wird \textbf{succ} (\enquote{successor}) genutzt um den Wert einer Variable zu erhöhen. Der Wert einer Variable kann beliebig oft erhöht werden, der kleinste Wert einer Variable ist jedoch immer null.

\begin{lstlisting}[language=c, caption=pred und succ in While, label={lst:while-var-defdec}]
	var r0 := 1; // Erzeugt eine Variable mit dem Namen 'r0' und dem Wert 1
	pred(r0); // Der Wert ist nun 0
	succ(r0); // Der Wert ist nun 1
\end{lstlisting}

\subsection{Loop und While}
\textbf{loop} und \textbf{while} sidn zwei unterschiedliche Schleifenarten, welche in der Programmiersprache while genutzt werden können. Der Schleifenkörper befindet sich in beiden Fällen einem \textbf{begin:} und \textbf{end}. Innerhalb eines Schleifenkörpers können beliebig viele Anweisungen oder auch andere Schleifen stehen. Variablen, welche innerhalb eines Schleifenkörpers definiert werden, existieren nur innerhalb vom Schleifenkörper.

Die \textbf{while} Schleife verhält sich, wie man es aus anderen Sprachen gewohnt ist: der Körper der Schleife wird wiederholt, solange der Wert der Variable im Schleifenkopf ungleich null ist. Mithilfe einer \textbf{while}-Schleife ist es möglich Endlosschleifen zu erzeugen.

\begin{lstlisting}[language=c, caption=while-Schleife in While, label={lst:while-var-defdec}]
	var r0 := 3; // Erzeugt eine Variable mit dem Namen 'r0' und dem Wert 3
	while(r0) begin: // Die Schleife laeuft solange r0 != 0
		pred(r0); // Veraender den Wert von r0
	end
\end{lstlisting}

Die \textbf{loop} Schleife hingegen verhält sich ähnlich zu einer for-Schleife aus anderen Programmiersprache. Die Variable im Schleifenkopf wird automatisch verringert bis der Wert gleich null ist. Aus diesem Grund kann davon ausgegangen werden, dass die Schleife terminiert. Um das sicher zu stellen, ist es bei einer \textbf{loop}-Schleife nicht möglich die Variable, welche im Schleifenkopf steht, schreibend zu verwenden. 

\begin{lstlisting}[language=c, caption=loop-Schleife in While, label={lst:while-var-defdec}]
	var r0 := 3; // Erzeugt eine Variable mit dem Namen 'r0' und dem Wert 3
	loop(r0) begin: // r0 wird bei jedem Durchlauf automatisch verringert
		var r1 := r0; // Der Wert von r0 darf gelesen werden.
		pred(r0); // Es ist innerhalb der Schleife verboten den Wert zu aendern!
	end
\end{lstlisting}

\subsection{Read und Write}


\subsection{Funktionen}