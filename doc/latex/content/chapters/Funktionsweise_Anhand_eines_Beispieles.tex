\chapter{Anleitungen}
Im folgenden Kapitel soll erklärt werden wie die Ergebnisse dieser Arbeit, der Compiler und die \ac{vsc} Erweiterung, verwendet werden.

\section{Die \acs{vsc} Erweiterung installieren}
Es gibt zwei unterschiedliche Möglichkeiten die \ac{vsc} Erweiterung runter zu laden und zu installieren.

\subsection{In der \acs{ide}} 
Der einfachste Weg ist die Installation, direkt aus der \ac{ide} heraus. Dazu muss zunächst das Erweiterungsfenester geöffnet werden, dies geschiet mit einem Klick auf die vier Quadrate an der Sidebar. Alternativ lässt sich das Fenster auch über \enquote{Anzeige} -> \enquote{Erweiterungen} öffnen.

Ist dies geschehen, wird dem Nutzer angeboten eine Erweiterung zu suchen. Es ist ausreichend \enquote{while} in die Suchleiste zu tippen um die Erweiterung zu finden. Die richtige Erweiterung kann an dem dunkelblauen Logo mit einem großen gelben \enquote{W} erkannt werden. Alternativ lässt sich auch nach \enquote{@id:ZitronenSchrank.while-language-extension} suchen, wobei die richtige Erweiterung die einzige sein wird, welche als Ergebnis aufgelistet wird. 

Nach der Auswahl des korrekten Suchergebnisses ist es dem Benutzer möglich die Erweiterung zu installieren. Eventuell muss die \ac{ide} anschließend neugestartet werden, darauf wird der Nutzer jedoch gegebenenfalls hingewiesen.

Falls der Nutzer nach der Installation der Erweiterung eine Datei mit der Endung \enquote{while} oder \enquote{loop} in \ac{vsc} öffnet, wird die Erweiterung automatisch aktiviert und der Programmcode entsprechend eingefärbt und der Nutzer beim Schreiben unterstützt.

\subsection{Mithilfe einer VSIX-Datei}
Eine andere und umständliche Möglichkeit die Erweiterung zu installieren ist mithilfe einer \enquote{VSIX}-Datei. Diese muss manuell heruntergeladen und installiert werden. Ein Ort, an welchem die \enquote{VSIX}-Datei heruntergeladen werden kann, ist das \href{https://github.com/ZitronenSchrank/While-Extension-VsCode}{Github-Reposity} der Erweiterung. Eine andere Möglichkeit ist der Download über den \href{https://marketplace.visualstudio.com/items?itemName=ZitronenSchrank.while-language-extension}{Visual Studio Marketplace}

Nachdem die Datei runter geladen wurde, lässt sie sich in \ac{vsc} installieren. Dazu muss zunächst das Erweiterungsfenster geöffnet werden und anschließend mit einem Klick auf das $\cdots$-Symbol die Option \enquote{Install From VSIX} ausgewählt und anschließend die heruntergeladene Datei ausgesucht werden. 


\section{Den Compiler verwenden}
Beim Compiler handelt es sich um eine Java-Jar-Datei, welche über die Kommandozeile ausgeführt werden muss, da keine \ac{gui} existiert. Dazu muss eine Eingabebauforderung im selben Ordner gestartet werden, in welchem sich auch der Compiler befindet. In \cref{lst:howto-basiccall} ist zu sehen wie der Compiler grundsätzlich aufgerufen wird.

\begin{lstlisting}[language=java, caption=Grundlegenes Aufrufen des Compilers, label={lst:howto-basiccall}]
	java -jar while-1.0.jar
\end{lstlisting}

Damit der Compiler eine ausführbare Datei erzeugen kann, muss als erster Parameter ein Pfad zu einer Quelltext-Datei angegeben werden, wie in \cref{lst:howto-basiccompile} exemplarisch dargestellt. 

\begin{lstlisting}[language=java, caption=Kompelieren einer Datei, label={lst:howto-basiccompile}]
	java -jar while-1.0.jar ./test.while
\end{lstlisting}

Nach dem Ausführen werden eventuelle Fehlermeldungen auf der Konsole angezeigt und bei Erfolg eine Class-Datei mit dem selben Namen im selben Ordner erzeugt, in welchem sich auch die Eingabedatei befindet. 

Es ist auch möglich den Compiler mit zwei Parametern aufzurufen, wie in \cref{lst:howto-compilewithoutput} zu sehen ist. Der erste Parameter ist wieder der zu kompilierende Quelltext und der zweite Parameter gibt an, welchen Namen das Kompilat tragen soll. 

\begin{lstlisting}[language=java, caption=Kompelieren einer Datei mit vorgegebener Ausgabe, label={lst:howto-compilewithoutput}]
	java -jar while-1.0.jar ./test.while ./hallo
\end{lstlisting}

Ein Kompilat lässt sich ausführen, wie man es bereits von Class-Dateien gewöhnt ist, in \cref{lst:howto-runCompile} ist zu sehen, wie eine test.class-Datei ausgeführt werden kann. 

\begin{lstlisting}[language=java, caption=Ein programm ausführen, label={lst:howto-runCompile}]
	java -cp . test
\end{lstlisting}

\section{Den Compiler kompilieren}

\begin{lstlisting}[language=java, caption=Mit Maven den Compiler compilieren, label={lst:howto-compileCompiler}]
	mvn clean compile assembly:single
\end{lstlisting}


\section{Den Compiler erweitern} 
\subsection{Grammatik anpassen}

\subsection{Semantische Überprüfung implementieren}

\subsection{Codegenerierung}