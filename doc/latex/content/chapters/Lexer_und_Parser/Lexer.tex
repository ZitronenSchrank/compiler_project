\section{Lexer} \label{sec:lexer}
Wird ein Programmcode vom Benutzer einem Compiler übergeben, landet dieser im ersten Schritt beim Lexer. Andere Namen für den Lexer sind auch \enquote{Tokenizer} oder \enquote{Scanner}. Die Aufgabe des Lexers ist es, die einzelnen Zeichen in einem Programmcode zu \textbf{Token} entsprechend einer Grammatik zusammen zu fügen. Ein Token ist zusammengehörige Menge von einzelnen Zeichen in einer Programmiersprache. Beispielsweise ist das Schlüsselwort \textbf{var} ein Token, genau so wie der Name einer Variable.

Stößt der Lexer beim Erzeugen der Token auf einen Fehler beispielsweise ein unbekanntes Zeichen, so erzeugt ein Lexer eine Fehlermeldung, welche dem Nutzer mitgeteilt wird.