\section{Parsergenerator}
\subsection{Prinzip}
Der Lexer und Parser müssen stets der selben Grammatik entsprechen, damit die beiden Bausteine zusammenarbeiten können. Das hat die Konsequenz, dass bei jeder Änderung der Grammatik auch der Lexer und Parser angepasst werden müssen, was einfache Anpassungen an der Grammatik sehr aufwendig macht.

Aus diesem Grund ist es üblich den Lexer und Parser anhand einer Grammatik von einem sogenannten Parsergenerator generieren zu lassen. Dadurch kann immer davon ausgegangen werden, dass Lexer und Parser immer derselben aktuellen Grammatik entsprechen. Ein weiterer Vorteil der Nutzung eines Parsergenerators ist es, dass Lexer und Parser für unterschiedliche Zielprogrammiersprachen generiert werden können. Beispielsweise wird für den in dieser Arbeit entwickelten Compiler ein Lexer und Parser als Java-Klassen generiert und für die IDE-Integration (\cref{chap:ide})  wurden TypeScript Klassen erzeugt.  

\subsection{ANTLR}
In dieser Arbeit wurde der Parsergenerator \ac{antlr} verwendet. Dabei handelt es sich um einen offenen Parsergenerator, welcher seit dem Jahr 1989 von Terence Parr entwickelt wird. \ac{antlr} wird in vielen großen Projekten eingesetzt, wie beispielsweise bei Twitter oder der Netbeans IDE. \cite{TerenceParr2022} Auf der \href{}{offiziellen Website} von \ac{antlr} ist eine Anleitung zum herunterladen und Installieren des Programms zu finden. 

\ac{antlr} generiert neben dem Lexer und Parser für eine Grammatik zusätzlich noch Klassen um den \ac{ast}, welcher vom Parser erzeugt wird, zu durchlaufen. Dafür werden zwei Möglichkeiten angeboten, welche jeweils einem Design-Pattern entsprechen: Listener und Visitor.

In dieser Arbeit wurde sich für den Visitor-Ansatz entschieden, da dieser es ermöglicht selbst zu bestimmen wie der \ac{ast} durchlaufen werden soll. Beispielsweise werden erst alle Funktionen durchlaufen und anschließend alle Statements außerhalb einer Funktion auch dann, wenn die Funktionen am Ende des Programmcodes stehen, da dieses Vorgehen die semantische Überprüfung (\cref{chap:semantic}) simplifiziert.

\ac{antlr} erzeugt beim Visitor-Ansatz eine \enquote{BaseVisitor}-Klasse, welche für jeden Knoten eine leere Visit-Methode mit einem Parameter implementiert, welcher Informationen zum entsprechenden Knoten beinhaltet. In dieser Arbeit wurde für jeden Knoten eine Visitor-Klasse erstellt, welche von dieser Baseklasse erbt und die entsprechende Visit-Methode implementiert. Mithilfe der Informationen aus dem Parameter können wiederum die Unterknoten durchlaufenen werden.



