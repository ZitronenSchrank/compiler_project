\chapter*{Abstract}\markboth{Abstract}{}
\addcontentsline{toc}{chapter}{Abstract}

Compiler und Programmiersprachen sind das tägliche Werkzeug von vielen Programmierern, doch scheinen nur wenige ein Verständnis dafür zu haben, was eine Programmiersprache genau ist und welche Schritte ein Compiler durchläuft, bis ein ausführbares Programm erzeugt wird. In der vorliegenden Arbeit werden die Bestandteile einer Programmiersprache und eines Compilers vorgestellt und Schritt für Schritt erläutert, indem eine Programmiersprache samt dazugehörigem Compiler entwickelt wird. Ebenfalls wird erläutert, wie die neue Programmiersprache in die IDE \enquote{Visual Studio Code} eingebunden wird.

Zunächst wird eine Grammatik für die Sprache definiert und anschließend mit dem Parsergenerator \enquote{ANTLR} ein dazugehöriger Lexer und Parser erzeugt, um sie im Compiler und in der IDE-Erweiterung zu nutzen. Anschließend wird der Nutzen der semantischen Analyse erläutert und erklärt, wie sie für die Sprache implementiert wird. Im nächsten Schritt wird die Erzeugung von Javabytecode mithilfe der Bibliothek ASM vorgestellt.

Das Ergebnis ist eine funktionstüchtige Programmiersprache mit einem einsatzfähigen Compiler und hilfreicher IDE-Integration, welche zur Lehre im Modul \enquote{Automaten Und Formale Sprachen} eingesetzt werden könnte.